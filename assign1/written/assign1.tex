
\documentclass[11pt]{article}

\input{../../tex/defs.tex}

% Useful syntax commands
\newcommand{\jarr}[1]{\left[#1\right]}   % \jarr{x: y} = {x: y}
\newcommand{\jobj}[1]{\left\{#1\right\}} % \jobj{1, 2} = [1, 2]
\newcommand{\pgt}[1]{\, > {#1}}          % \pgt{1} = > 1
\newcommand{\plt}[1]{\, < {#1}}          % \plt{2} = < 2
\newcommand{\peq}[1]{\, = {#1}}          % \peq{3} = = 3
\newcommand{\prop}[1]{\langle{#1}\rangle}% \prop{x} = <x>
\newcommand{\matches}[2]{{#1}\sim{#2}}   % \matches{a}{b} = a ~ b
\newcommand{\aeps}{\varepsilon}          % \apes = epsilon
\newcommand{\akey}[2]{.{#1}\,{#2}}       % \akey{s}{a} = .s a
\newcommand{\aidx}[2]{[#1]\,{#2}}        % \aidx{i}{a} = [i] a
\newcommand{\apipe}[1]{\mid {#1}}        % \apipe{a} = | a

\newcommand{\conj}[2]{#1 \wedge #2}
\newcommand{\disj}[2]{#1 \vee #2}

% Other useful syntax commands:
%
% \msf{x} = x (not italicised)
% \falset = false
% \truet = true
% \tnum = num
% \tbool = bool
% \tstr = str


\begin{document}

\hwtitle
{Assignment 1}
{Will Crichton (wcrichto)} %% REPLACE THIS WITH YOUR NAME/ID

\problem{Problem 1}

Part 1:

\begin{alignat*}{1}
  \msf{Property}~p ::= \qamp \varepsilon     \\
  \mid \qamp \plt{n}                         \\
  \mid \qamp \pgt{n}                         \\
  \mid \qamp \peq{n}                         \\
  \mid \qamp \conj{p}{p}                     \\
  \mid \qamp \disj{p}{p}                     \\
  \\
  \msf{Schema}~\tau ::= \qamp \tnum \prop{p} \\
  \mid \qamp \tstr \prop{p}                  \\
  \mid \qamp \tbool                          \\
  \mid \qamp \jarr{\tau}                     \\
  \mid \qamp \jobj{(s : \tau)^*}             \\
\end{alignat*}

Part 2:

% mathpar is the environment for writing inference rules. It takes care of
% the spacing and line breaks automatically. Use "\\" in the premises to
% space out multiple assumptions.
\begin{mathpar}

  \ir{S-Bool-False}{\ }{\matches{\falset}{\tbool}}

  \ir{S-Bool-True}{\ }{\matches{\truet}{\tbool}}

  % Inference rules go here

  \ir{P-Empty}{\ }{\matches{j}{\varepsilon}}

  \ir{P-And}{\matches{j}{p_0} \\
    \matches{j}{p_1}}
  {\matches{j}{\conj{p_0}{p_1}}}

  \ir{P-Or0}{\matches{j}{p_0}}{\matches{j}{\disj{p_0}{p_1}}}

  \ir{P-Or1}{\matches{j}{p_1}}{\matches{j}{\disj{p_0}{p_1}}}

  \ir{P-Lt}{nj \plt{nt}}{\matches{nj}{\plt{nt}}}

  \ir{P-Gt}{nj \pgt{nt}}{\matches{nj}{\pgt{nt}}}

  \ir{P-Eq}{nj \peq{nt}}{\matches{nj}{\peq{nt}}}

  \ir{S-Number}{\matches{n}{p}}{\matches{n}{\tnum \prop{p}}}

  \ir{S-String}{\matches{s}{p}}{\matches{s}{\tstr \prop{p}}}

  \ir{S-Arr}
  {\forall i = 0 \ldots |j| - 1 \ . \ \matches{j_i}{\tau}}
  {\matches{\jarr{j^*}}{{\jarr{\tau}}}}

  \ir{S-Obj}
  {\forall s \in st \ . \ \conj{s \in sj}{\matches{j_s}{\tau_s}}}
  {\matches{\jobj{(sj : j)^*}}{\jobj{(st : \tau)^*}}}

\end{mathpar}

\problem{Problem 2}

Part 1:

\begin{mathpar}
  % Inference rules go here
  \ir{E-Key}{\ }{\steps{(\akey{s}{a}, \jobj{s : j, \ldots})}{(a, j)}}

  \ir{E-Index}{\ }{\steps{(\aidx{i}{a}, \jarr{\ldots, j_i, \ldots})}{(a, j_i)}}

  \ir{E-Pipe-Elem}
  {\forall i = 0 \ldots |j| - 1 \ . \ \steps{(a, j_i)}{(a', j_i')}}
  {\steps{(\apipe{a}, \jarr{j^*})}{(\apipe{a'}, \jarr{j'^*})}}

  \ir{E-Pipe-Empty}
  {\ }{\steps{(\apipe{\eps}, \jarr{j^*})}{(\eps, \jarr{j^*})}}

\end{mathpar}

Part 2:

\begin{mathpar}
  \ir{A-Empty}{\ }{\matches{\eps}{\tau}}

  \ir{A-Key}{\matches{a}{\tau}}{\matches{\akey{k}{a}}{\jobj{k: \tau, \ldots}}}

  \ir{A-Index}{\matches{a}{\tau}}{\matches{\aidx{i}{a}}{\jarr{\tau}}}

  \ir{A-Pipe}{\matches{a}{\tau}}{\matches{\apipe{a}}{\jarr{\tau}}}

\end{mathpar}

\textit{Accessor safety}: for all $a, j, \tau$, if $\matches{a}{\tau}$ and $\matches{j}{\tau}$, then there exists a $j'$ such that $\evals{(a, j)}{\aeps, j'}$.

\begin{proof}
  % Proof goes here.
\end{proof}

\end{document}
